Zunächst wurde die Entwicklung mit Phidata ausprobiert. Diese erwies sich jedoch im Nachhinein als nicht zuverlässig genug für unsere Anforderungen. 
Aus diesem Grund folgt im nächsten Abschnitt eine Implementierung mit Langchain.


\subsection{Phidata Implementation}
\label{sec:phidata-implementation}

Das folgende Python-Codebeispiel zeigt, wie die \texttt{Agent}-Klasse mit dem \texttt{SqlAgentStorage}-Backend kombiniert wird:

\begin{lstlisting}[language=Python, style=pythonstyle, caption={Phidata Implementation des KI-Agenten}]
from phi.agent import Agent
from phi.model.ollama import Ollama
from phi.storage.agent.sqlite import SqlAgentStorage

# Erstellen eines Speicherbackends mit einer Sqlite-Datenbank
storage = SqlAgentStorage(
    table_name="product_data (product_number INTEGER PRIMARY KEY, input_voltage INTEGER, input_current INTEGER, output_voltage INTEGER, output_current INTEGER, number_io_ports INTEGER, bus_protocol TEXT)",
    db_file="poc/db/data.db",
)

# Initialisieren des Agents mit dem Ollama-Modell und dem Speicher
agent = Agent(
    model=Ollama(id="deepseek-r1:14b"),
    markdown=True,
    storage=storage
)

# Ausgabe der Antwort im Terminal
agent.print_response("Which input voltage does the product with the product number 2 have? Take a look in your database first.")
\end{lstlisting}

\subsubsection{Ausgabe}

\begin{figure}[H]
    \begin{center}
        \includegraphics[width=1\linewidth]{Figures/results/phidata_01.png} 
        \captionof{figure}{Beispielausgabe 1}
        \label{fig:phidata-bsp01}
    \end {center}
\end{figure}

\begin{figure}[H]
    \begin{center}
        \includegraphics[width=1\linewidth]{Figures/results/phidata_02.png} 
        \captionof{figure}{Beispielausgabe 2}
        \label{fig:phidata-bsp02}
    \end {center}
\end{figure}

\subsection{Erstellung der Datenbank}

Falls die Sqlite-Datenbank noch nicht existiert, muss sie erstellt werden. Der folgende Code stellt sicher, dass die Datenbank und die Tabelle existieren, und fügt einige Beispieldaten hinzu.

\begin{lstlisting}[language=Python, style=pythonstyle, caption={Code zur Erstellung der Beispiel-Datenbank}]
import sqlite3
import os

# Pfad zur Datenbankdatei
db_path = './poc/db/produkte.db'

# Überprüfen, ob die Datenbankdatei existiert
if not os.path.exists(db_path):
    # Erstellen der Datenbankdatei
    open(db_path, 'w').close()

    # Verbindung zur SQLite-Datenbank herstellen (wird erstellt, wenn sie nicht existiert)
    conn = sqlite3.connect(db_path)
    cursor = conn.cursor()

    # Tabelle erstellen
    cursor.execute('''
    CREATE TABLE IF NOT EXISTS produkte (
        product_number INTEGER PRIMARY KEY,
        input_voltage INTEGER,
        input_current INTEGER,
        output_voltage INTEGER,
        output_current INTEGER,
        number_io_ports INTEGER,
        bus_protocol TEXT
    )
    ''')

    # Beispiel-Daten hinzufügen
    cursor.execute('''
    INSERT INTO produkte (product_number, input_voltage, input_current, output_voltage, output_current, number_io_ports, bus_protocol)
    VALUES
    (1, 24, 2, 24, 2, 4, 'Modbus TCP'),
    (2, 12, 1, 24, 2, 2, 'Modbus RTU'),
    (3, 24, 2, 12, 1, 4, 'Modbus TCP')
    ''')

    # Änderungen speichern und Verbindung schließen
    conn.commit()
    conn.close()
else:
    print("Datenbank existiert bereits.")
\end{lstlisting}

Dieser Code stellt sicher, dass die Datenbankdatei existiert, und erstellt gegebenenfalls eine neue Datenbank mit einer Tabelle für Produktdaten. Es werden Beispieldaten in die Tabelle eingefügt, wenn die Datenbank neu erstellt wird.

\subsection{Offizielle Dokumentation}

\subsubsection{Sqlite Agent Storage}

Phidata unterstützt die Verwendung von Sqlite als Speicher-Backend für Agents über die \texttt{SqlAgentStorage}-Klasse. \cite{phidata_sqliteagent}

\textbf{Verwendung:}

Es ist erforderlich, entweder \texttt{db\_url}, \texttt{db\_file} oder \texttt{db\_engine} anzugeben. Das folgende Beispiel verwendet \texttt{db\_file}:

\begin{lstlisting}[language=Python, style=pythonstyle, caption={Verwendung von \texttt{SqlAgentStorage}}]
from phi.storage.agent.sqlite import SqlAgentStorage

# Erstellen eines Speicherbackends mit einer Sqlite-Datenbank
storage = SqlAgentStorage(
    table_name="agent_sessions",     # Speichert Sitzungen in der ai.sessions-Tabelle
    db_file="tmp/data.db",           # Sqlite-Datenbankdatei
)

# Speicher zum Agent hinzufügen
agent = Agent(storage=storage)
\end{lstlisting}

\subsubsection{Parameter}

Der Konstruktor von \texttt{SqlAgentStorage} nimmt die folgenden Parameter entgegen:

\begin{table}[h]
    \caption{Parameter des \texttt{SqlAgentStorage}-Konstruktors}
    \centering
    \begin{tabular}{|p{3cm}|p{3cm}|p{3cm}|p{6cm}|}
        \hline
        \textbf{Parameter}        & \textbf{Typ}      & \textbf{Standardwert} & \textbf{Beschreibung}                                        \\
        \hline
        \texttt{table\_name}      & string             & -                & Name der zu verwendenden Tabelle.                              \\
        \texttt{schema}           & Optional string    & "ai"             & Name des Schemas, Standardwert ist "ai".                        \\
        \texttt{db\_url}          & Optional string    & None             & Datenbank-URL, falls angegeben.                                \\
        \texttt{db\_engine}       & Optional \texttt{Engine} & None         & Zu verwendender Datenbank-Engine.                              \\
        \texttt{schema\_version}  & integer            & 1                & Version des Schemas, Standardwert ist 1.                        \\
        \texttt{auto\_}\texttt{upgrade\_} \texttt{schema} & boolean        & False            & Wenn wahr, wird das Schema automatisch aktualisiert, wenn erforderlich. \\
        \hline
    \end{tabular}
\end{table}

\subsection{Langchain Implementation}
\label{sec:langchain-implementation}

Dieser Python-Code zeigt, wie man mit LangChain und SQLite eine einfache Agentenanwendung aufbaut. Der Agent kann auf Datenbankanfragen reagieren und auch Aktualisierungen in der SQLite-Datenbank durchführen. Die Anwendung nutzt das \texttt{ChatOllama}-Modell, um natürliche Sprachabfragen zu verarbeiten und entsprechende SQL-Abfragen zu generieren.

\subsubsection{Code-Erklärung}

\minisec{Imports}
Im ersten Abschnitt werden die notwendigen Bibliotheken importiert:
\begin{itemize}
    \item \texttt{sqlite3}: Für die Interaktion mit der SQLite-Datenbank.
    \item \texttt{langchain.chat\_models.ChatOllama}: Um das Ollama-Modell für die Verarbeitung von Benutzeranfragen zu nutzen.
    \item \texttt{langchain.agents}: Enthält Funktionen zum Initialisieren des LangChain-Agenten.
    \item \texttt{langchain.tools}: Bietet Werkzeuge wie \texttt{Tool} und \texttt{QuerySQLDataBaseTool}, die dem Agenten helfen, SQL-Abfragen auszuführen.
    \item \texttt{langchain.memory.ConversationBufferMemory}: Speichert den Verlauf der Konversation, um den Kontext beizubehalten.
    \item \texttt{langchain.sql\_database.tool.QuerySQLDataBaseTool}: Ein Tool, das für die Ausführung von SQL-Abfragen auf einer SQLite-Datenbank verwendet wird.
    \item \texttt{langchain.sql\_database.SQLDatabase}: Eine Abstraktionsebene für den Zugriff auf eine SQLite-Datenbank.
\end{itemize}

\minisec{Datenbankverbindung}
Der Code verbindet sich mit einer SQLite-Datenbank, die sich im Verzeichnis \texttt{./poc/db/produkte.db} befindet. Hierbei wird das \texttt{sqlite3}-Modul verwendet, um eine Verbindung zur Datenbank herzustellen.

\begin{lstlisting}[language=Python]
db_path = "./poc/db/produkte.db"  # Path zur SQLite-Datenbank
conn = sqlite3.connect(db_path)
cursor = conn.cursor()
database = SQLDatabase.from_uri(f"sqlite:///{db_path}")
\end{lstlisting}

\minisec{Query-Tool und Update-Tool}
Es werden zwei Tools definiert:
\begin{itemize}
    \item \textbf{QuerySQLiteDB}: Dieses Tool verarbeitet SQL-Abfragen, die der Agent basierend auf den Benutzereingaben erstellt. Es nutzt die \texttt{QuerySQLDataBaseTool}-Klasse von LangChain.
    \item \textbf{ExecuteUpdateQuery}: Ein weiteres Tool, das SQL-UPDATE-Abfragen ausführt, um die Datenbank basierend auf den Benutzereingaben zu aktualisieren.
\end{itemize}

\begin{lstlisting}[language=Python]
query_tool = QuerySQLDataBaseTool(db=database)

def execute_update_query(update_sql):
    """Führt eine SQL-UPDATE-Abfrage aus."""
    try:
        cursor.execute(update_sql)
        conn.commit()
        return "Datenbank erfolgreich aktualisiert."
    except sqlite3.Error as e:
        return f"Datenbankaktualisierung fehlgeschlagen: {str(e)}"
\end{lstlisting}

\minisec{Initialisierung des Ollama-Modells}
Das Ollama-Modell wird initialisiert, um die Benutzereingaben zu verarbeiten und auf die Datenbank zuzugreifen. Hierbei wird die \texttt{ChatOllama}-Klasse verwendet:

\begin{lstlisting}[language=Python]
llm = ChatOllama(model="llama3.1:latest")  # Ollama-Modell initialisieren
\end{lstlisting}

\minisec{Speicher für die Konversation}
Der \texttt{ConversationBufferMemory}-Speicher sorgt dafür, dass der Agent den Verlauf der Konversation behält und somit auf vorherige Anfragen reagieren kann.

\begin{lstlisting}[language=Python]
memory = ConversationBufferMemory(memory_key="chat_history", return_messages=True)
\end{lstlisting}

\minisec{Werkzeuge für den Agenten}
Die beiden zuvor definierten Tools \texttt{QuerySQLiteDB} und \texttt{ExecuteUpdateQuery} werden als Werkzeuge für den Agenten konfiguriert:

\begin{lstlisting}[language=Python]
tools = [
    Tool(
        name="QuerySQLiteDB",
        func=query_tool.run,
        description="Verwenden Sie dieses Tool, um Informationen aus der SQLite-Datenbank abzurufen."
    ),
    Tool(
        name="ExecuteUpdateQuery",
        func=execute_update_query,
        description="Verwenden Sie dieses Tool, um ein Update auf der Datenbank auszuführen."
    )
]
\end{lstlisting}

\minisec{Initialisierung des Agenten}
Der Agent wird mit der \texttt{initialize\_agent}-Funktion initialisiert. Der Agent verwendet den \texttt{ZERO\_SHOT\_REACT\_DESCRIPTION}-Agententyp und die konfigurierten Werkzeuge. Zudem wird die Konversationshistorie gespeichert.

\begin{lstlisting}[language=Python]
agent = initialize_agent(
    agent=AgentType.ZERO_SHOT_REACT_DESCRIPTION,
    tools=tools,
    llm=llm,
    verbose=True,
    memory=memory
)
\end{lstlisting}

\minisec{Chat-Schleife}
Der Agent läuft in einer Endlosschleife, in der er Benutzereingaben verarbeitet und darauf reagiert. Die Schleife endet, wenn der Benutzer \texttt{exit} eingibt.

\begin{lstlisting}[language=Python]
print("Chat with the LLM (type 'exit' to stop)")
while True:
    user_input = input("You: ")
    if user_input.lower() == "exit":
        break
    response = agent.run(user_input)
    print("LLM:", response)
\end{lstlisting}

\subsection{Beispielausgaben}

\begin{figure}[H]
    \begin{center}
        \includegraphics[width=1\linewidth]{Figures/results/langchain_AgentExecutor-chain_Read.png} 
        \caption{Lesen der Datenbank, Aufruf: "Which input-voltage does the product with product-number 2 have?"}
        \label{fig:langchain-bsp01}
    \end {center}
\end{figure}

\begin{figure}[H]
    \begin{center}
        \includegraphics[width=1\linewidth]{Figures/results/langchain_AgentExecutor-chain_Update.png}
        \includegraphics[width=1\linewidth]{Figures/results/langchain_thoughts_Update.png} 
        \caption{Aktualisieren der Datenbank, Aufruf: "Update the input-voltage of the product with product-number 2 to 15V."}
        \label{fig:langchain-bsp02}
    \end {center}
\end{figure}